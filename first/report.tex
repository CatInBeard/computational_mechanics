\documentclass[a4paper]{article}
\usepackage[14pt]{extsizes} % для того чтобы задать нестандартный 14-ый размер шрифта
\usepackage[utf8]{inputenc}
\usepackage[russian]{babel}
\usepackage{setspace,amsmath}
\usepackage{csvsimple}
\usepackage{caption}
\usepackage{graphicx}
\usepackage{float}

\usepackage[dvipsnames]{xcolor}
\usepackage{fancyvrb}

\graphicspath{{img/}}
\DeclareGraphicsExtensions{.pdf,.png,.jpg}
\usepackage[left=20mm, top=15mm, right=15mm, bottom=15mm, nohead, footskip=10mm]{geometry} % настройки полей документа
 
\begin{document} % начало документа
 
\begin{titlepage}

\thispagestyle{empty}

\centerline{Санкт-Петербургский политехнический университет}
\centerline{Высшая школа теоретической механики, ФизМех}

\vfill
\centerline{Направление подготовки}
\centerline{<<01.03.03\_2 Биомеханика и медицинская инженерия>>}

\vfill

\centerline{Индивидуальное задание №1}
\begin{center}
    Тема <<Метод конечных разностей. Нестационарное уравнение\\ теплопроводности.>>
\end{center}    
\centerline{Дисциплина <<Вычислительная механика>>}
\centerline{Вариант 9}

\vfill

Выполнила студент гр. 5030103/90201: \hfill Ефимов Г.М. 

Преподаватель: \hfill  Витохин Е.Ю.

\vfill

\centerline{Санкт-Петербург}
\centerline{\the\year}
\clearpage
\end{titlepage}

\newpage
     
    \tableofcontents
\newpage
 
\newpage
\section{Постановка задачи}
    \addtocontents{toc}{\protect\setcounter{tocdepth}{1}}
    Составить решение смешанной задачи для дифференциального уравнения параболического типа $\frac{\partial u}{\partial t} = \frac{\partial^{2} u}{\partial x^{2}}$ (уравнение теплопроводности) используя метод конечных разностей при заданных начальных и граничных условиях:\\
    \begin{center}
    $u(x,0)=sin(x)+0.08$\\
    $u(0,t)=0.08+2 t$\\
    $u(0.6,t)=0.6446$\\
    $x \in [0,0.6]$, $t \in [0,0.01]$
    \end{center}
    \newpage
\section{Явная схема}
    \addtocontents{toc}{\protect\setcounter{tocdepth}{2}}  
    \subsection{Уравнения}
        Уравнение теплопроводности имеет вид:
        \begin{equation}\label{eq:parabol_eq}
            \rho C_v \frac{\partial T}{\partial t}=\lambda \frac{\partial^{2} T}{\partial x^{2}}
        \end{equation}
        Разложим $T(x,t)$ в окрестности точки $x_0$ в ряд Тейлора:
        \begin{equation}\label{eq:taylor_p}
            T(x_0+h)=T(x_0)+T^{\prime}(x_0) \frac{h}{1!}+T^{\prime \prime}(x_0) \frac{h^{2}}{2!}+o(h^2)
        \end{equation}
        \begin{equation}\label{eq:taylor_m} 
            T(x_0-h)=T(x_0)-T^{\prime}(x_0) \frac{h}{1!}+T^{\prime \prime}(x_0) \frac{h^{2}}{2!}+o(h^2)
        \end{equation}
        Складываем \ref{eq:taylor_p} и \ref{eq:taylor_m} уравнения и выражаем $ T^{\prime \prime}(x_0) $:
        \begin{equation}\label{eq:taylor_sym} 
            T^{\prime \prime}(x_0)=\frac{T(x_0+h)-2 T(x_0) T(x_0-h)}{h^2}=\frac{T^{k}_{i+1}-2 T^{k}_{i}+T^{k}_{i-1}}{h^2}
        \end{equation}
        Разложим $T(x,t)$ в окрестности точки $t_0$ в ряд Тейлора:
        \begin{equation}\label{eq:taylor_p_t} 
            T(t_0+\Delta t)=T(t_0)+\dot{T}(t_0) \frac{\Delta t}{1!}+o(\Delta t)
        \end{equation}
        \begin{equation}\label{eq:taylor_m_t} 
            T(t_0-\Delta t)=T(t_0)-\dot{T}(t_0) \frac{\Delta t}{1!}+o(\Delta t)
        \end{equation}
        Складываем \ref{eq:taylor_p_t} и \ref{eq:taylor_m_t} уравнения и выражаем $ \dot{T}(t_0) $:
        \begin{equation}\label{eq:taylor_sym_t} 
            \dot{T}(t_0)=\frac{T(t_0-+Delta t)-T(t_0)}{\Delta t}=\frac{T^{k+1}_{i}+T^{k}_{i}}{\Delta t}
        \end{equation}
        Конечно-разностное уравнение теплопроводности примет вид:
        \begin{equation}\label{eq:mkr} 
        \rho C_v \frac{T^{k+1}_{i}+T^{k}_{i}}{\Delta t} = \lambda \frac{T^{k}_{i+1}-2 T^{k}_{i}+T^{k}_{i-1}}{h^2}
        \end{equation}
        Таким образом, явная схема метода имеет вид:
        \begin{equation}\label{eq:explicit_schema} 
            T^{k+1}_{i}=\frac{\Delta t \lambda}{\rho h^2 C_v}(T^{k}_{i+1}-2 T^{k}_{i}+T^{k}_{i-1})+T^{k}_{i}
        \end{equation}
        В текущей задаче примем $\rho \equiv 1$, $C_v \equiv 1$,$\lambda \equiv 1$
        \newpage
    \subsection{Результаты вычислений}
        \begin{tabular}{l c}
        & x\\
        t &
        \csvautotabular{"data/data1_latex.csv"}
        \end{tabular}
        \captionof{table}{Результаты вычислений температуры при помощи явной схемы.}
        \begin{figure}[h]
            \includegraphics[width=0.6\linewidth]{1}
            \caption{Явная схема. Модель распределения температур относительно пространства и времени.}
            \label{ris:explicit_3d}
        \end{figure}
        \begin{figure}[h]
            \includegraphics[width=0.6\linewidth]{4}
            \caption{Явная схема. Изменение температуры в различные моменты времени.}
            \label{ris:explicit_2d}
        \end{figure}
        \newpage
\section{Неявная схема}
    \subsection{Формулы}
        Запишем конечно-разностные выражения производных:
        \begin{equation}\label{eq:implicit_schema_T_t}
            \frac{\partial T}{\partial t}=\frac{T^{k+1}_{i}-T^{k}_{i}}{\Delta t}
        \end{equation}
        \begin{equation}\label{eq:implicit_schema_T_x2}
            \frac{\partial^{2} T}{\partial x^{2}}=\frac{T^{k+1}_{i-1}-2 T^{k+1}_{i} +T^{k+1}_{i+1}}{h^2}
        \end{equation}
        Подставим \ref{eq:implicit_schema_T_t} и \ref{eq:implicit_schema_T_x2} в уравнение теплопроводности \ref{eq:parabol_eq} и получим уравнение неявной схемы:
        \begin{equation}\label{eq:implicit_Fi}
            -A T^{k+1}_{i-1}+B T^{K+1}_{i}- C T^{k+1}_{i+1}=F_{i}
        \end{equation}
        \begin{equation}
            A=C=\frac{\lambda}{h^2}
        \end{equation}
        \begin{equation}
            B=\frac{\rho C_v h^{2}+ 2 \lambda \Delta t}{\Delta t h^2}
        \end{equation}
        \begin{equation}
            F_i=\frac{\rho C_v}{\Delta t} T^{k}_{i}
        \end{equation}
        В текущей задаче примем $\rho \equiv 1$, $C_v \equiv 1$,$\lambda \equiv 1$
        Применим метод прогонки для решения уравнения \ref{eq:implicit_Fi}:
        \subsubsection*{Прямой ход:}
            \begin{equation}
                s_1=\frac{C}{B}
            \end{equation}
            \begin{equation}
                m_1=\frac{F^{k}_{i}}{B}
            \end{equation}
            \begin{equation}
                s_i=\frac{C}{-A s_{i-1}+B}
            \end{equation}
            \begin{equation}
                m_i=\frac{F^{k}_i+A m_{i-1}}{B-A s_{i-1}}
            \end{equation}
        \subsubsection*{Обратный ход:}
            \begin{equation}
                T^{k+1}_i=s_i T^{k+1}_{i+1}+m_i
            \end{equation}
        \newpage
    \subsection{Результаты вычислений}
        \begin{tabular}{l c}
            & x\\
            t &
            \csvautotabular{"data/data2_latex.csv"}
        \end{tabular}
        \captionof{table}{Результаты вычислений температуры при помощи неявной схемы.}
        \begin{figure}[h]
            \includegraphics[width=0.6\linewidth]{2}
            \caption{Неявная схема. Модель распределения температур относительно пространства и времени.}
            \label{ris:implicit_3d}
        \end{figure}
        \begin{figure}[h]
            \includegraphics[width=0.6\linewidth]{5}
            \caption{Неявная схема. Изменение температуры в различные моменты времени.}
            \label{ris:implicit_2d}
        \end{figure}
        \newpage
    \section{Сравнение явной и неявной схемы}
        \addtocontents{toc}{\protect\setcounter{tocdepth}{1}}
        \begin{figure}[h]
            \includegraphics[width=0.6\linewidth]{3}
            \caption{Сравнение срезов явной и неявной схемы в разные моменты времени}
            \label{ris:diff_2d}
        \end{figure}
        По рисунку \ref{ris:diff_2d} можно сделать вывод, что явная схема даёт более гладкое решение, чем неявная.
        \newpage
    \section{Исходный код программы}
        \addtocontents{toc}{\protect\setcounter{tocdepth}{1}}
        \textbf{main.cpp:}
        \VerbatimInput{main.cpp}
\end{document}