\documentclass[a4paper]{article}
\usepackage[14pt]{extsizes} % для того чтобы задать нестандартный 14-ый размер шрифта
\usepackage[utf8]{inputenc}
\usepackage[russian]{babel}
\usepackage{setspace,amsmath}
\usepackage{csvsimple}
\usepackage{caption}
\usepackage{graphicx}
\usepackage{float}

\usepackage[dvipsnames]{xcolor}
\usepackage{fancyvrb}

\graphicspath{{img/}}
\DeclareGraphicsExtensions{.pdf,.png,.jpg}
\usepackage[left=20mm, top=15mm, right=15mm, bottom=15mm, nohead, footskip=10mm]{geometry} % настройки полей документа
 
\begin{document} % начало документа
 
\begin{titlepage}

\thispagestyle{empty}

\centerline{Санкт-Петербургский политехнический университет}
\centerline{Высшая школа теоретической механики, ФизМех}

\vfill
\centerline{Направление подготовки}
\centerline{<<01.03.03\_2 Биомеханика и медицинская инженерия>>}

\vfill

\centerline{Индивидуальное задание №2}
\begin{center}
    Тема <<Метод конечных разностей. Уравнение колебания\\ струны.>>
\end{center}    
\centerline{Дисциплина <<Вычислительная механика>>}
\centerline{Вариант 9}

\vfill

Выполнила студент гр. 5030103/90201: \hfill Ефимов Г.М. 

Преподаватель: \hfill  Витохин Е.Ю.

\vfill

\centerline{Санкт-Петербург}
\centerline{\the\year}
\clearpage
\end{titlepage}

\newpage
     
    \tableofcontents
\newpage
\section{Постановка задачи}
    \addtocontents{toc}{\protect\setcounter{tocdepth}{1}}
    Составить решение смешанной задачи для дифференциального уравнения колебания струны $\frac{\partial^{2} u}{\partial t^2} = \frac{\partial^{2} u}{\partial x^{2}}$ используя метод конечных разностей при заданных начальных и граничных условиях:\\
    \begin{center}
    $u(x,0)=x ( 2 x - 0.5 )$\\
    $u_{t}(x,0)=cos(2 x)$\\
    $u(0,t)=t^{2}$\\
    $u(0.6,t)=1.5$\\
    $x \in [0,1]$, $t \in [0,0.5]$
    \end{center}
    \newpage
\section{Явная схема}
    \addtocontents{toc}{\protect\setcounter{tocdepth}{2}}  
    \subsection{Уравнения}
        Уравнение теплопроводности имеет вид:
        \begin{equation}\label{eq:parabol_eq}
            \frac{\partial^2 u}{\partial x^2} - \frac{1}{c^{2}} \frac{\partial^2 u}{\partial t^2}=0
        \end{equation}
        Где $c$ --- это скорость распространения звуковых волн
        Разложим $U(x,t)$ в окрестности точки $x_0$ в ряд Тейлора:
        \begin{equation}\label{eq:taylor_p}
            U(x_0+h)=U(x_0)+U^{\prime}(x_0) \frac{h}{1!}+U^{\prime \prime}(x_0) \frac{h^{2}}{2!}+o(h^2)
        \end{equation}
        \begin{equation}\label{eq:taylor_m} 
            U(x_0-h)=U(x_0)-U^{\prime}(x_0) \frac{h}{1!}+U^{\prime \prime}(x_0) \frac{h^{2}}{2!}+o(h^2)
        \end{equation}
        Складываем \ref{eq:taylor_p} и \ref{eq:taylor_m} уравнения и выражаем $ U^{\prime \prime}(x_0) $:
        \begin{equation}\label{eq:taylor_sym} 
            U^{\prime \prime}(x_0)=\frac{U(x_0+h)-2 U(x_0) U(x_0-h)}{h^2}=\frac{U^{k}_{i+1}-2 U^{k}_{i}+U^{k}_{i-1}}{h^2}
        \end{equation}
        Разложим $U(x,t)$ в окрестности точки $t_0$ в ряд Тейлора:
        \begin{equation}\label{eq:taylor_p_t} 
            U(t_0+\Delta t)=U(t_0)+\dot{U}(t_0) \frac{\Delta t}{1!}+\ddot{U}(t_0) \frac{\Delta t^2}{1!}+o(\Delta t^{2})
        \end{equation}
        \begin{equation}\label{eq:taylor_m_t} 
            U(t_0-\Delta t)=U(t_0)-\dot{U}(t_0) \frac{\Delta t}{1!}+\ddot{U}(t_0) \frac{\Delta t^2}{1!}+o(\Delta t^{2})
        \end{equation}
        Складываем \ref{eq:taylor_p_t} и \ref{eq:taylor_m_t} уравнения и выражаем $ \dot{U}(t_0) $:
        \begin{equation}\label{eq:taylor_sym_t} 
            \ddot{U}(t_0)=\frac{U(t_o-\Delta t)-2 U(t_0)+U(t_0+\Delta t)}{\Delta t^2}
        \end{equation}
        Введём сетки для пространства $x=i*h$ и времени $t=j \Delta t$. Тогда выражения \ref{eq:taylor_sym} и \ref{eq:taylor_sym_t} примут вид: 
        \begin{equation}\label{eq:mkr} 
            U^{\prime \prime} = \frac{U^{k}_{i-1}-2 U^{k}_{i}+U^{k}_{i+1}}{h^2}
        \end{equation}
        \begin{equation}\label{eq:mkr_dt} 
            \ddot{U} = \frac{U^{k-1}_{i}-2 U^{k}_{i}+U^{k+1}_{i}}{t^2}
        \end{equation}
        Подставим выражения \ref{eq:mkr}, \ref{eq:mkr_dt} в \ref{eq:parabol_eq} и получим конечно-разностное уравнение:
        \begin{equation}
            \frac{U^{k}_{i-1}-2 U^{k}_{i}+U^{k}_{i+1}}{h^2}-\frac{1}{c^2} \frac{U^{k-1}_{i}-2 U^{k}_{i}+U^{k+1}_{i}}{t^2}
        \end{equation}
        Таким образом, явная схема метода имеет вид:
        \begin{equation}\label{eq:explicit_schema} 
            U^{k+1}_{i}=\frac{c^2 \Delta t^{2}}{h^2}(U^{k}_{i-1}-2 U^{k}_{i}+U^{k}_{i+1})+2 U^{k}_{i}-U^{k-1}_{i}
        \end{equation}
        Чтобы применить схему, необходимо заполнить второй слой по времени. Для этого рассмотрим формулу:
        \begin{equation}\label{eq:U_x_t}
            U(0+\Delta t)=U(0)+\dot{U}(0) \frac{\Delta t}{1!}+\ddot{U}(0) \frac{\Delta t^2}{2!}+o(\Delta t^2)
        \end{equation}
        По формуле \ref{eq:parabol_eq}
        \begin{equation}\label{eq:U_dd_t}
            \ddot{U}(0)=c^2 U^{\prime \prime}(0)=\frac{c^{2}}{h^{2}}(U^{0}_{i-1}-2 U^{0}_{i}+U^{0}_{i+1})
        \end{equation}
        Используя формулу \ref{eq:U_x_t}, формула \ref{eq:U_dd_t} принимает вид:
        \begin{equation}\label{eq:U_dd_t}
            U^{1}_{i}= U^{0}_{i}+U^{0}_{t_{i}}+\frac{c^{2} \Delta t^2}{2 h^{2}}(U^{0}_{i-1}-2 U^{0}_{i}+U^{0}_{i+1})
        \end{equation}             
        В текущей задаче примем $c=1,h=0.1, \Delta t=0.045$
        \newpage
    \subsection{Результаты вычислений}
        \tiny
        \begin{tabular}{l c}
        & x\\
        t &
        \csvautotabular{"data/data1_latex.csv"}
        \end{tabular}
        \normalsize
        \captionof{table}{Результаты вычислений поперечного перемещения колебаний струны при помощи явной схемы.}
        \begin{figure}[h]
            \includegraphics[width=0.6\linewidth]{1}
            \caption{Явная схема. Модель поперечного перемещения колебаний струны.}
            \label{ris:explicit_3d}
        \end{figure}
        \begin{figure}[h]
            \includegraphics[width=0.6\linewidth]{4}
            \caption{Явная схема. Изменение температуры в различные моменты времени.}
            \label{ris:explicit_2d}
        \end{figure}
        \newpage
\section{Неявная схема}
    \subsection{Формулы}
        Запишем конечно-разностные выражения производных:
        \begin{equation}\label{eq:implicit_schema_U_t}
            \frac{\partial^{2} U}{\partial t^{2}}=\frac{U^{k-1}_{i+1}-2 U^{k}_{i+1}-U^{k+1}_{i}}{\Delta t}
        \end{equation}
        \begin{equation}\label{eq:implicit_schema_U_x2}
            \frac{\partial^{2} U}{\partial x^{2}}=\frac{U^{k+1}_{i-1}-2 U^{k+1}_{i} +U^{k+1}_{i+1}}{h^2}
        \end{equation}
        Подставим \ref{eq:implicit_schema_U_t} и \ref{eq:implicit_schema_U_x2} в уравнение теплопроводности \ref{eq:parabol_eq} и получим уравнение неявной схемы:
        \begin{equation}\label{eq:implicit_Fi}
            -A U^{k+1}_{i-1}+B U^{K+1}_{i}- C U^{k+1}_{i+1}=F_{i}
        \end{equation}
        \begin{equation}
            A=C=\frac{1}{h^2}
        \end{equation}
        \begin{equation}
            B=\frac{h^{2}+ 2 \Delta t^2}{c^{2} \Delta t^2 h^2}
        \end{equation}
        \begin{equation}
            F_i=\frac{2 U^{k}_{i} - U^{k-1}_{i}}{\Delta t^2 C^2}
        \end{equation}
        В текущей задаче примем $c=1,h=0.1, \Delta t=0.045$
        Применим метод прогонки для решения уравнения \ref{eq:implicit_Fi}:
        \subsubsection*{Прямой ход:}
            \begin{equation}
                s_1=\frac{C}{B}
            \end{equation}
            \begin{equation}
                m_1=\frac{F^{k}_{i}}{B}
            \end{equation}
            \begin{equation}
                s_i=\frac{C}{B-A s_{i-1}}
            \end{equation}
            \begin{equation}
                m_i=\frac{F^{k}_i+A m_{i-1}}{B-A s_{i-1}}
            \end{equation}
        \subsubsection*{Обратный ход:}
            \begin{equation}
                U^{k+1}_i=s_i U^{k+1}_{i+1}+m_i
            \end{equation}
        \newpage
    \subsection{Результаты вычислений}
        \tiny
        \begin{tabular}{l c}
            & x\\
            t &
            \csvautotabular{"data/data2_latex.csv"}
        \end{tabular}
        \normalsize
        \captionof{table}{Результаты вычислений поперечного перемещения колебаний струны при помощи неявной схемы.}
        \begin{figure}[h]
            \includegraphics[width=0.6\linewidth]{2}
            \caption{Неявная схема. Модель поперечного перемещения колебаний струны.}
            \label{ris:implicit_3d}
        \end{figure}
        \begin{figure}[h]
            \includegraphics[width=0.6\linewidth]{5}
            \caption{Неявная схема. Перемещения в различные моменты времени.}
            \label{ris:implicit_2d}
        \end{figure}
        \newpage
    \section{Сравнение явной и неявной схемы}
        \addtocontents{toc}{\protect\setcounter{tocdepth}{1}}
        \begin{figure}[h]
            \includegraphics[width=0.6\linewidth]{3}
            \caption{Сравнение срезов явной и неявной схемы в разные моменты времени.}
            \label{ris:diff_2d}
        \end{figure}
        %По рисунку \ref{ris:diff_2d} можно сделать вывод, что явная схема даёт более гладкое решение, чем неявная.
        \newpage
    \section{Исходный код программы}
        \addtocontents{toc}{\protect\setcounter{tocdepth}{1}}
        \textbf{main.cpp:}
        \VerbatimInput{main.cpp}
\end{document}